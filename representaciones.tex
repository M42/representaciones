%%%%%%%%%%%%%%%%%%%%%%%%%%%%%%%%%%%%%%%%%%%%%%%%%%%%%%%%%%%%%%%%%%%%%%%%%%%%%%%%%%%%%%%%%%%%%%%%%%%%%%
% Plantilla básica de Latex en Español.
%
% Autor: Andrés Herrera Poyatos (https://github.com/andreshp)
%
% Es una plantilla básica para redactar documentos. Utiliza el paquete fancyhdr para darle un
% estilo moderno pero serio.
%
% La plantilla se encuentra adaptada al español.
%
%%%%%%%%%%%%%%%%%%%%%%%%%%%%%%%%%%%%%%%%%%%%%%%%%%%%%%%%%%%%%%%%%%%%%%%%%%%%%%%%%%%%%%%%%%%%%%%%%%%%%%

%-----------------------------------------------------------------------------------------------------
%	INCLUSIÓN DE PAQUETES BÁSICOS
%-----------------------------------------------------------------------------------------------------

\documentclass{article}

% Lenguaje
\usepackage[spanish,es-noquoting, es-tabla, es-lcroman]{babel}
\usepackage[utf8]{inputenc}
\selectlanguage{spanish}


% Matemáticas
\usepackage{amsthm}
\usepackage{amsfonts}
\usepackage{amsmath}
\usepackage{tikz-cd}
\theoremstyle{plain}
\newtheorem{theorem}{Teorema}
\newtheorem{proposition}{Proposición}
\newtheorem{lemma}{Lema}
\newtheorem{corollary}{Corolario}
\theoremstyle{definition}
\newtheorem{definition}{Definición}
\theoremstyle{remark}
\newtheorem*{remark}{Nota}
\renewcommand*{\proofname}{Demostración}


% Fuente
\usepackage{courier}
\usepackage{microtype}


% ----
% Estilo de página
%----
% Paquetes para el diseño de página:
\usepackage{fancyhdr}               % Utilizado para hacer títulos propios.
\usepackage{lastpage}               % Referencia a la última página. Utilizado para el pie de página.
\usepackage{extramarks}             % Marcas extras. Utilizado en pie de página y cabecera.
\usepackage[parfill]{parskip}       % Crea una nueva línea entre párrafos.
\usepackage{geometry}               % Asigna la "geometría" de las páginas.

% Se elige el estilo fancy y márgenes de 3 centímetros.
\pagestyle{fancy}
\geometry{left=3cm,right=3cm,top=3cm,bottom=3cm,headheight=1cm,headsep=0.5cm} % Márgenes y cabecera.
% Se limpia la cabecera y el pie de página para poder rehacerlos luego.
\fancyhf{}

% Espacios en el documento:
\linespread{1.1}                        % Espacio entre líneas.
\setlength\parindent{0pt}               % Selecciona la indentación para cada inicio de párrafo.

% Cabecera del documento. Se ajusta la línea de la cabecera.
\renewcommand\headrule{
	\begin{minipage}{1\textwidth}
	    \hrule width \hsize
	\end{minipage}
}

% Texto de la cabecera:
\lhead{\docauthor}                          % Parte izquierda.
\chead{}                                    % Centro.
\rhead{\subject \ - \doctitle}              % Parte derecha.

% Pie de página del documento. Se ajusta la línea del pie de página.
\renewcommand\footrule{
\begin{minipage}{1\textwidth}
    \hrule width \hsize
\end{minipage}\par
}

\lfoot{}                                                 % Parte izquierda.
\cfoot{}                                                 % Centro.
\rfoot{Página\ \thepage\ de\ \protect\pageref{LastPage}} % Parte derecha.


% ----
% PORTADA
% ----

% Estilo
\usepackage{title1}

% Título y autores
\newcommand{\doctitle}{Aplicaciones de caracteres}
\newcommand{\docsubtitle}{}
\newcommand{\docdate}{1 \ de \ Enero \ de \ 2015}
\newcommand{\subject}{Representaciones}
\newcommand{\docauthor}{D. Charte, J.C. Entrena, L. Soto, M. Román}
\newcommand{\docaddress}{Universidad de Granada}
\newcommand{\docemail}{}

% Resumen
\newcommand{\docabstract}{El cálculo de orientaciones en el espacio tridimensional es necesario para la generación de gráficos y animaciones por ordenador. Los ángulos de Euler representan el espacio de orientaciones pero no forman un homeomorfismo local, lo que causa errores gráficos (bloqueo del cardán), mientras que los cuaterniones de Hamilton sí proporcionan un recubrimiento de dos hojas del mismo. Con ellos presentaremos un ejemplo de movimiento de objetos en gráficos tridimensionales.}

\begin{document}

\maketitle

% Profundidad del Índice
%\setcounter{tocdepth}{1}
\newpage
\tableofcontents
\newpage

\section{El grupo de rotación}
\subsection{El grupo de rotación SO(3)}

\begin{definition}
  Llamamos $O(n)$ al grupo de las \textbf{matrices ortogonales}
  de dimensiones $n \times n$, aquellas que cumplen que $Q^TQ = QQ^T = I$,
  bajo la composición.
\end{definition}

Nótese que las matrices ortogonales forman un subgrupo del grupo
lineal $GL(n)$ de matrices invertibles; y que, por definición, sólo
pueden tener determinante $1$ y $-1$.

\begin{definition}
  Llamamos $SO(n)$ al \textbf{subgrupo de rotaciones}, definido como
  el subgrupo de $O(n)$ formado por aquellas matrices que tienen
  determinante $1$.
\end{definition}

Vamos a considerar $SO(3)$, las rotaciones sobre el origen de $\mathbb R^3$, pues son laz que nos interesan en este caso.
Vamos a ver un resultado sobre la topología de este grupo.

\begin{proposition}
	$SO(3)$ es homeomorfo a $\mathbb{RP}^3$
\end{proposition}

\begin{proofname}\\
	Consideramos una esfera en $\mathbb R^3$ de radio $\pi$, $\mathbb B = B(0, \pi)$. Asignamos a cada punto $x \in \mathbb B$
	una rotación, con eje la recta que une el origen y $x$, y ángulo de rotación $\alpha = d(x, O)$ la distancia con signo al
	origen. De esta forma, cada	punto representa una única rotación, salvo los puntos antípoda, a los que corresponden las
	rotaciones de ángulos $\pi$ y $-\pi$, que son la misma. Esto nos lleva a identificar los puntos antípoda de $\partial \mathbb B$,
	lo que establece un homeomorfismo entre $SO(3)$ y $\mathbb{RP}^3$. $\qed$
\end{proofname}


\begin{remark}
	De hecho, por ser $\mathbb B$ una variedad diferenciable $\mathbb{C^{\infty}}$, el homeomorfismo es un difeomorfismo.
\end{remark}

De este resultado se deduce que $SO(3)$ es conexo, pero no es simplemente conexo, pues su grupo fundamental no es trivial.



En cada rotación existe una recta invariante, cuyos puntos se mantienen fijos tras aplicarse la rotación.
Llamaremos a esta recta el eje de rotación, y vamos a ver que siempre existe.


\begin{proposition}
	Cada rotación de $SO(3)$ deja fija una recta.
\end{proposition}

\begin{proofname}\\
	Sea $ M \in SO(3)$ una rotación. Queremos ver que la matriz tiene al $1$ como valor propio, que nos dará un subespacio
	de al menos dimensión 1 cuyos puntos se mantienen fijos por $M$.

	Que el $1$ sea valor propio equivale a que $ det(M - I_3) = 0$. Desarrollamos la expresión sabiendo que $M^t = M^{-1}$

	\[ det(M - I_3) = det((M - I_3)^t) ) det(M^t - I_3) = det(M^{-1} - I_3) = \]
	\[= det(M^{-1}(I_3 - M)) = det(M^{-1}) * det(I_3 - M) = det(I_3 - M) = -det(M - I_3) \]

	de donde se deduce directamente que $det(M - I_3) = 0$ y que el $1$ es un autovalor de M. $\qed$
\end{proofname}


\section{Ángulos de Euler}
\subsection{Definición}

Los ángulos de Euler son tres ángulos utilizados para describir la
orientación de un objeto respecto a un sistema coordenadas fijo en el
espacio. Análogamente, podemos no considerar como fijo el sistema de
coordenadas y ver los ángulos de Euler como una orientación de dicho
sistema.

La forma tradicional de usar rotaciones en el espacio se basa en el
uso de las llamadas rotaciones de Euler, que combinan hasta tres
rotaciones sobre los ejes cartesianos, donde la única restricción es
que no puede haber dos rotaciones sobre el mismo eje contiguas, pues
derivarían en una.

Las tres rotaciones sobre los ejes cartesianos se pueden combinar de
12 formas distintas, consistentes en las 6 ordenaciones posibles
usando los tres índices, además de aquellas seis en las que aparece el
mismo eje al principio y al final, y uno de los dos restantes en la
posición central.


\subsection{Descomposición}
Cada rotación se descompone en sus tres ángulos de Euler de forma
única. Vemos las matrices de rotación respecto a los ejes cartesianos,
de forma que la primera es la rotación de ángulo $\psi$ sobre el eje
$z$, la segunda rota $\phi$ grados sobre el eje $y$, y la tercera la
que rota $\theta$ grados con respecto al eje $x$. Las tres rotaciones
están consideradas en sentido antihorario.
\footnote{Para cambiar el sentido no hay más que cambiarle el signo a los senos}

\[\begin{pmatrix}
    \cos \psi & -\sin \psi & 0 \\
    \sin \psi & \cos \psi & 0 \\
    0 & 0 & 1
  \end{pmatrix}\begin{pmatrix}
      \cos \phi & 0 & \sin \phi \\
      0 & 1 & 0 \\
      -\sin \phi & 0 & \cos \phi
    \end{pmatrix}\begin{pmatrix}
    1 & 0 & 0 \\
    0 & \cos \theta & -\sin \theta \\
    0 & \sin \theta & \cos \theta
  \end{pmatrix}
\]

Podemos ver el espacio de los ángulos de Euler como el 3-toro
$\mathbb T^3$, sin más que identificar cada variable de giro en una
dimensión del toro.

% https://es.wikipedia.org/wiki/Teorema_de_rotaci%C3%B3n_de_Euler

\subsection{Bloqueo del cardán}

El principal problema del uso de los ángulos de Euler es el llamado
\textit{bloqueo del cardán}. Este fenómeno consiste en la pérdida de
un grado de libertad a la hora de rotar, debida a la alineación de dos
de los ejes de rotación. Esto ocurre cuando se dan ciertas
combinaciones en los ángulos de rotación, provocando dicho
alineamiento.

Formalmente, el bloqueo del cardán ocurre debido a que el espacio de
ángulos de Euler, representados con $\mathbb T^3$, no puede ser
recubridor del espacio de rotaciones, que se identifica con
$\mathbb{RP}^3$. Esto puede verse fácilmente tomando los grupos
fundamentales de ambos espacios. Mientras que el grupo fundamental del
3-toro es $\pi(\mathbb T^3, x) \simeq \mathbb Z^3$, el grupo
fundamental del 3-espacio proyectivo es
$\pi(\mathbb{RP}^3, x) \simeq \mathbb Z_2$. Como el grupo fundamental
de un recubridor debe ser subgrupo del grupo fundamental del espacio
reducido, deducimos que no puede haber un recubrimiento. De hecho,
como $Z_2$ no tiene subgrupos propios, los posibles espacios
recubridores deben tener grupo fundamental $Z_2$ o ser simplemente
conexos.

\section{Cuaternios}
\subsection{Versores}
Sabemos que los cuaternios pueden escribirse como
$\mathbb{H} = \mathbb{R} \oplus V$, donde $V$ es el espacio vectorial
real tridimensional de base $i,j,k$, siendo equivalente a la
definición

\[
  V = \left\{ v \in \mathbb{H} \mid v^2 < 0 \right\}
\]

Los cuaternios unitarios, en particular, pueden escribirse como
$\cos \theta + u \sin \theta$, donde $u$ es un vector unitario de
$\mathbb{R}^3$. A un cuaternio así escrito se le llama
\textbf{versor}.

\begin{proof}
  Si tenemos $q = a + bv$ con $v$ unitario, sabemos que $v^2 = - v(-v) = - vv^\ast = 1$.
  Entonces, si $q$ es unitario,

  \[
    1 = qq^\ast = (a+bv)(a-bv) = a^2 + b^2.
  \]

  Con lo que existirá algún $\theta$ para el que $a = \cos \theta$ y $b = \sin \theta$.
\end{proof}

\subsection{Conexión con SU(2)}
\cite{gelfand63}
% https://en.wikipedia.org/wiki/Rotation_group_SO(3)#Connection_between_SO.283.29_and_SU.282.29

\begin{definition}
  Llamamos $\mathrm{SU}(n)$, al espacio de las matrices complejas unitarias de determinante $1$.
  Es decir,

  \[SU(n) = \left\{ M \in U_n(\mathbb{C}) \mid \mathrm{det}(M) = 1 \right\}.\]
\end{definition}

En particular nos interesamos por $SU(2)$ porque podemos verlo como un embebimiento de la
esfera tridimensional en $\mathbb{R}^4$.

\begin{theorem}
\label{su2}
  Cada matriz en $SU(2)$ puede escribirse como

  \[\begin{pmatrix}
      a+bi & c+di \\
      -c+di & a-bi
    \end{pmatrix}\]

  donde $a,b,c,d \in \mathbb{R}$ y $a^2+b^2+c^2+d^2 = 1$.
\end{theorem}
\begin{proof}
  Si tomamos $\alpha,\beta,\gamma,\delta \in \mathbb{C}$ formando una matriz unitaria de
  determinante $1$ como

    \[\begin{pmatrix}
      \alpha & \beta \\
      \gamma & \delta
    \end{pmatrix},\]

  tenemos las relaciones siguientes, obtenidas del hecho de que debe ser ortogonal

  \begin{itemize}
    \item $|\alpha|+|\beta| = 1$
    \item $\alpha\delta-\beta\gamma = 1$
    \item $\alpha\overline{\gamma} + \beta\overline{\delta} = 0$
  \end{itemize}

  y que se resuelven con $\alpha = \overline{\delta}$ y
  $\beta = -\overline{\gamma}$; mientras que la primera condición
  nos da la condición $a^2+b^2+c^2+d^2 = 1$.
\end{proof}


\subsection{Cuaternios como esfera de dimensión 4}

Tomando la condición obtenida anteriormente, $a^2+b^2+c^2+d^2 = 1$, podemos ver los cuaternios de $SU(2)$ como elementos $x \in
\mathbb R^4$ que verifican $\left|x\right| = 1 \implies SU(2) \cong S^3$. La aplicación definida en el teorema \ref{su2} nos da
el isomorfismo entre los dos espacios, que de hecho es un difeomorfismo, sin más que considerar el difeomorfismo entre $M_2(\mathbb C)$
y $R^8$ y restringir.

\subsection{Recubrimiento del grupo de rotaciones}

A diferencia del caso de los ángulos de Euler, los cuaternios unitales
sí son un recubrimiento del espacio de rotaciones $SO(3)$. Esto se
deduce directamente del hecho de que $S^3$ es un recubridor de dos
hojas de $\mathbb{RP}^3$ (de hecho, es su recubridor universal).

<<<<<<< HEAD
Al ser recubridor de dos hojas, tenemos que los cuaternios que se
identifican por la relación de equivalencia de $\mathbb{RP}^3$
representan la misma rotación, pues van al mismo elemento por la
aplicación recubridora. En este caso, son los cuaternios opuestos.
=======
Al ser recubridor de dos hojas, tenemos que los cuaternios que se identifican por la relación de equivalencia de $\mathbb{RP}^3$
representan la misma rotación, pues van al mismo elemento por la aplicación recubridora. En este caso, son los cuaternios opuestos.
>>>>>>> c5c09c44c7561749964ae5875014aadb7e17f768

\section{Aplicación en gráficos}

\subsection{Exponenciación de cuaternios}
\begin{definition}
  Se define la función \textbf{exponencial en cuaternios},
  $\exp \colon \mathbb{H} \to \mathbb{H}$, desde la serie de potencias
  de la función exponencial usual, es decir,

  \[
    \exp(q) =  1 + q + \frac{q^2}{2!} + \frac{q^3}{3!} + \dots + \frac{q^n}{n!} + \dots .
  \]
\end{definition}

Si escribimos los cuaternios como versores,
$q = \cos \theta + v \sin \theta$, tendremos que como $v^2 = -1$, se
cumple la fórmula de Euler para cualquier $\theta \in \mathbb{R}$ como
$\exp(\theta v) = \cos \theta + v \sin \theta = q$, y por tanto
podemos calcular la exponente de un cuaternio como

\[
  q^t = \exp(t \theta v) = \cos (t \theta) + v \sin (t \theta).
\]

\subsection{Spherical Linear Interpolation (SLERP)}
El método \textbf{SLERP} (Spherical Lineal Interpolation) permite
interpolar un punto entre dos orientaciones de manera continua. Dadas
dos orientaciones $q_1,q_2$ representadas como cuaterniones, un punto
$p$ y un parámetro de interpolación $t$, buscamos un camino continuo
que interpole $p$ desde $q_1$, cuando $t=0$, hasta $q_2$, cuando
$t=1$.

Para calcular la rotación interpolada, usamos la fórmula

\[q' = q_1(q_1^{-1}q_2)^t\]

Cuando interpolamos usando SLERP, la interpolación de cuaternios
induce a su vez una interpolación en las rotaciones tridimensionales,
que nos dará una rotación con velocidad angular uniforme y alrededor
de un eje fijo.
% Por qué? https://en.wikipedia.org/wiki/Slerp#Quaternion_Slerp

Sin embargo, como el recubrimiento es doble, el camino de rotaciones
puede ser doble y ser o bien el camino más corto, o bien el camino más
largo.

% http://number-none.com/product/Understanding%20Slerp,%20Then%20Not%20Using%20It/
% https://www.geometrictools.com/Documentation/Quaternions.pdf

\subsection{Spherical and Quadrangle (SQUAD)}

Sean $a,b,p,q$ cuatro cuaternios, vistos como vértices entre los que
realizaremos la interpolación cúbica. La idea es interpolar un punto
$c$ entre $p$ y $q$ mientras que, a la vez, interpolamos un $d$ entre
$a$ y $b$; finalmente, interpolamos con una fórmula cuadrática entre
los puntos de las interpolaciones previas $d$ y $c$.

\[
  \mathrm{Squad}(p,q,a,b;t) =
  \mathrm{Slerp}(\mathrm{Slerp}(p,q;t), \mathrm{Slerp}(a,b;t); 2t(1-t))
\]

% http://run.usc.edu/cs520-s13/assign2/p245-shoemake.pdf

\section{Referencias}
% https://en.wikipedia.org/wiki/Euler_angles
% https://en.wikipedia.org/wiki/Gimbal_lock
% https://en.wikipedia.org/wiki/Rotation_group_SO(3)
% https://en.wikipedia.org/wiki/Plate_trick
% https://en.wikipedia.org/wiki/Charts_on_SO(3)
% https://www.3dgep.com/understanding-quaternions/

\begin{thebibliography}{9}

\bibitem{gelfand63}
  Gelfand, I.M.; Minlos, R.A.; Shapiro, Z.Ya. (1963),
  Representations of the Rotation and Lorentz Groups and their Applications,
  New York: Pergamon Press.

\end{thebibliography}

\end{document}
