%%%%%%%%%%%%%%%%%%%%%%%%%%%%%%%%%%%%%%%%%%%%%%%%%%%%%%%%%%%%%%%%%%%%%%%%%%%%%%%%%%%%%%%%%%%%%%%%%%%%%%
% Plantilla básica de Latex en Español.
%
% Autor: Andrés Herrera Poyatos (https://github.com/andreshp)
%
% Es una plantilla básica para redactar documentos. Utiliza el paquete fancyhdr para darle un
% estilo moderno pero serio.
%
% La plantilla se encuentra adaptada al español.
%
%%%%%%%%%%%%%%%%%%%%%%%%%%%%%%%%%%%%%%%%%%%%%%%%%%%%%%%%%%%%%%%%%%%%%%%%%%%%%%%%%%%%%%%%%%%%%%%%%%%%%%

%-----------------------------------------------------------------------------------------------------
%	INCLUSIÓN DE PAQUETES BÁSICOS
%-----------------------------------------------------------------------------------------------------

\documentclass{article}

% Lenguaje
\usepackage[spanish,es-noquoting, es-tabla, es-lcroman]{babel}
\usepackage[utf8]{inputenc}
\selectlanguage{spanish}


% Matemáticas
\usepackage{amsthm}
\usepackage{amsmath}
\usepackage{tikz-cd}
\theoremstyle{plain}
\newtheorem{theorem}{Teorema}
\newtheorem{proposition}{Proposición}
\newtheorem{lemma}{Lema}
\newtheorem{corollary}{Corolario}
\theoremstyle{definition}
\newtheorem{definition}{Definición}
\theoremstyle{remark}
\newtheorem*{remark}{Nota}
\renewcommand*{\proofname}{Demostración}


% Fuente
\usepackage{courier}
\usepackage{microtype}


% ----
% Estilo de página
%----
% Paquetes para el diseño de página:
\usepackage{fancyhdr}               % Utilizado para hacer títulos propios.
\usepackage{lastpage}               % Referencia a la última página. Utilizado para el pie de página.
\usepackage{extramarks}             % Marcas extras. Utilizado en pie de página y cabecera.
\usepackage[parfill]{parskip}       % Crea una nueva línea entre párrafos.
\usepackage{geometry}               % Asigna la "geometría" de las páginas.

% Se elige el estilo fancy y márgenes de 3 centímetros.
\pagestyle{fancy}
\geometry{left=3cm,right=3cm,top=3cm,bottom=3cm,headheight=1cm,headsep=0.5cm} % Márgenes y cabecera.
% Se limpia la cabecera y el pie de página para poder rehacerlos luego.
\fancyhf{}

% Espacios en el documento:
\linespread{1.1}                        % Espacio entre líneas.
\setlength\parindent{0pt}               % Selecciona la indentación para cada inicio de párrafo.

% Cabecera del documento. Se ajusta la línea de la cabecera.
\renewcommand\headrule{
	\begin{minipage}{1\textwidth}
	    \hrule width \hsize
	\end{minipage}
}

% Texto de la cabecera:
\lhead{\docauthor}                          % Parte izquierda.
\chead{}                                    % Centro.
\rhead{\subject \ - \doctitle}              % Parte derecha.

% Pie de página del documento. Se ajusta la línea del pie de página.
\renewcommand\footrule{
\begin{minipage}{1\textwidth}
    \hrule width \hsize
\end{minipage}\par
}

\lfoot{}                                                 % Parte izquierda.
\cfoot{}                                                 % Centro.
\rfoot{Página\ \thepage\ de\ \protect\pageref{LastPage}} % Parte derecha.


% ----
% PORTADA
% ----

% Estilo
\usepackage{title1}

% Título y autores
\newcommand{\doctitle}{Aplicaciones de caracteres}
\newcommand{\docsubtitle}{}
\newcommand{\docdate}{1 \ de \ Enero \ de \ 2015}
\newcommand{\subject}{Representaciones}
\newcommand{\docauthor}{D. Charte, J.C. Entrena, L. Soto, M. Román}
\newcommand{\docaddress}{Universidad de Granada}
\newcommand{\docemail}{}

% Resumen
\newcommand{\docabstract}{En este texto puedes incluir un resumen del documento. Este informa al lector sobre el contenido del texto, indicando el objetivo del mismo y qué se puede aprender de él.}

\begin{document}

\maketitle

% Profundidad del Índice
%\setcounter{tocdepth}{1}
\newpage
\tableofcontents
\newpage


\section{Cuaternios}

\subsection{El grupo de rotación SO(3)}

\begin{definition}
  Llamamos $O(n)$ al grupo de las \textbf{matrices ortogonales}
  de dimensiones $n \times n$, aquellas que cumplen que $Q^TQ = QQ^T = I$,
  bajo la composición.
\end{definition}

Nótese que las matrices ortogonales forman un subgrupo del grupo
lineal $GL(n)$ de matrices invertibles; y que, por definición, sólo
pueden tener determinante $1$ y $-1$.

\begin{definition}
  Llamamos $SO(n)$ al \textbf{subgrupo de rotaciones}, definido como
  el subgrupo de $O(n)$ formado por aquellas matrices que tienen
  determinante $1$.
\end{definition}


% DESARROLLAR:
% Cada rotación deja fija una recta

% DESARROLLLAR:
% SO(3), como grupo de Lie, es difeomorfo a RP^3.

\subsection{Conexión con SU(2)}
\cite{gelfand63}
% https://en.wikipedia.org/wiki/Rotation_group_SO(3)#Connection_between_SO.283.29_and_SU.282.29

\subsection{Ángulos de Euler}
Cada rotación se descompone únicamente en tres ángulos de Euler:

\[\begin{pmatrix}
    \cos \psi & \sin \psi & 0 \\
    -\sin \psi & \cos \psi & 0 \\
    0 & 0 & 1
  \end{pmatrix}\begin{pmatrix}
    1 & 0 & 0 \\
    0 & \cos \theta & \sin \theta \\
    0 & -\sin \theta & \cos \theta
  \end{pmatrix}\begin{pmatrix}
    \cos \phi & \sin \phi & 0 \\
    -\sin \phi & \cos \phi & 0 \\
    0 & 0 & 1
  \end{pmatrix}
\]



\section{Referencias}
% https://en.wikipedia.org/wiki/Euler_angles
% https://en.wikipedia.org/wiki/Gimbal_lock
% https://en.wikipedia.org/wiki/Rotation_group_SO(3)
% https://en.wikipedia.org/wiki/Plate_trick
% https://en.wikipedia.org/wiki/Charts_on_SO(3)


\begin{thebibliography}{9}

\bibitem{gelfand63}
  Gelfand, I.M.; Minlos, R.A.; Shapiro, Z.Ya. (1963),
  Representations of the Rotation and Lorentz Groups and their Applications,
  New York: Pergamon Press.

\end{thebibliography}

\end{document}
